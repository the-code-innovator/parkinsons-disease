
% Default to the notebook output style

    


% Inherit from the specified cell style.




    
\documentclass[11pt]{article}

    
    
    \usepackage[T1]{fontenc}
    % Nicer default font (+ math font) than Computer Modern for most use cases
    \usepackage{mathpazo}

    % Basic figure setup, for now with no caption control since it's done
    % automatically by Pandoc (which extracts ![](path) syntax from Markdown).
    \usepackage{graphicx}
    % We will generate all images so they have a width \maxwidth. This means
    % that they will get their normal width if they fit onto the page, but
    % are scaled down if they would overflow the margins.
    \makeatletter
    \def\maxwidth{\ifdim\Gin@nat@width>\linewidth\linewidth
    \else\Gin@nat@width\fi}
    \makeatother
    \let\Oldincludegraphics\includegraphics
    % Set max figure width to be 80% of text width, for now hardcoded.
    \renewcommand{\includegraphics}[1]{\Oldincludegraphics[width=.8\maxwidth]{#1}}
    % Ensure that by default, figures have no caption (until we provide a
    % proper Figure object with a Caption API and a way to capture that
    % in the conversion process - todo).
    \usepackage{caption}
    \DeclareCaptionLabelFormat{nolabel}{}
    \captionsetup{labelformat=nolabel}

    \usepackage{adjustbox} % Used to constrain images to a maximum size 
    \usepackage{xcolor} % Allow colors to be defined
    \usepackage{enumerate} % Needed for markdown enumerations to work
    \usepackage{geometry} % Used to adjust the document margins
    \usepackage{amsmath} % Equations
    \usepackage{amssymb} % Equations
    \usepackage{textcomp} % defines textquotesingle
    % Hack from http://tex.stackexchange.com/a/47451/13684:
    \AtBeginDocument{%
        \def\PYZsq{\textquotesingle}% Upright quotes in Pygmentized code
    }
    \usepackage{upquote} % Upright quotes for verbatim code
    \usepackage{eurosym} % defines \euro
    \usepackage[mathletters]{ucs} % Extended unicode (utf-8) support
    \usepackage[utf8x]{inputenc} % Allow utf-8 characters in the tex document
    \usepackage{fancyvrb} % verbatim replacement that allows latex
    \usepackage{grffile} % extends the file name processing of package graphics 
                         % to support a larger range 
    % The hyperref package gives us a pdf with properly built
    % internal navigation ('pdf bookmarks' for the table of contents,
    % internal cross-reference links, web links for URLs, etc.)
    \usepackage{hyperref}
    \usepackage{longtable} % longtable support required by pandoc >1.10
    \usepackage{booktabs}  % table support for pandoc > 1.12.2
    \usepackage[inline]{enumitem} % IRkernel/repr support (it uses the enumerate* environment)
    \usepackage[normalem]{ulem} % ulem is needed to support strikethroughs (\sout)
                                % normalem makes italics be italics, not underlines
    

    
    
    % Colors for the hyperref package
    \definecolor{urlcolor}{rgb}{0,.145,.698}
    \definecolor{linkcolor}{rgb}{.71,0.21,0.01}
    \definecolor{citecolor}{rgb}{.12,.54,.11}

    % ANSI colors
    \definecolor{ansi-black}{HTML}{3E424D}
    \definecolor{ansi-black-intense}{HTML}{282C36}
    \definecolor{ansi-red}{HTML}{E75C58}
    \definecolor{ansi-red-intense}{HTML}{B22B31}
    \definecolor{ansi-green}{HTML}{00A250}
    \definecolor{ansi-green-intense}{HTML}{007427}
    \definecolor{ansi-yellow}{HTML}{DDB62B}
    \definecolor{ansi-yellow-intense}{HTML}{B27D12}
    \definecolor{ansi-blue}{HTML}{208FFB}
    \definecolor{ansi-blue-intense}{HTML}{0065CA}
    \definecolor{ansi-magenta}{HTML}{D160C4}
    \definecolor{ansi-magenta-intense}{HTML}{A03196}
    \definecolor{ansi-cyan}{HTML}{60C6C8}
    \definecolor{ansi-cyan-intense}{HTML}{258F8F}
    \definecolor{ansi-white}{HTML}{C5C1B4}
    \definecolor{ansi-white-intense}{HTML}{A1A6B2}

    % commands and environments needed by pandoc snippets
    % extracted from the output of `pandoc -s`
    \providecommand{\tightlist}{%
      \setlength{\itemsep}{0pt}\setlength{\parskip}{0pt}}
    \DefineVerbatimEnvironment{Highlighting}{Verbatim}{commandchars=\\\{\}}
    % Add ',fontsize=\small' for more characters per line
    \newenvironment{Shaded}{}{}
    \newcommand{\KeywordTok}[1]{\textcolor[rgb]{0.00,0.44,0.13}{\textbf{{#1}}}}
    \newcommand{\DataTypeTok}[1]{\textcolor[rgb]{0.56,0.13,0.00}{{#1}}}
    \newcommand{\DecValTok}[1]{\textcolor[rgb]{0.25,0.63,0.44}{{#1}}}
    \newcommand{\BaseNTok}[1]{\textcolor[rgb]{0.25,0.63,0.44}{{#1}}}
    \newcommand{\FloatTok}[1]{\textcolor[rgb]{0.25,0.63,0.44}{{#1}}}
    \newcommand{\CharTok}[1]{\textcolor[rgb]{0.25,0.44,0.63}{{#1}}}
    \newcommand{\StringTok}[1]{\textcolor[rgb]{0.25,0.44,0.63}{{#1}}}
    \newcommand{\CommentTok}[1]{\textcolor[rgb]{0.38,0.63,0.69}{\textit{{#1}}}}
    \newcommand{\OtherTok}[1]{\textcolor[rgb]{0.00,0.44,0.13}{{#1}}}
    \newcommand{\AlertTok}[1]{\textcolor[rgb]{1.00,0.00,0.00}{\textbf{{#1}}}}
    \newcommand{\FunctionTok}[1]{\textcolor[rgb]{0.02,0.16,0.49}{{#1}}}
    \newcommand{\RegionMarkerTok}[1]{{#1}}
    \newcommand{\ErrorTok}[1]{\textcolor[rgb]{1.00,0.00,0.00}{\textbf{{#1}}}}
    \newcommand{\NormalTok}[1]{{#1}}
    
    % Additional commands for more recent versions of Pandoc
    \newcommand{\ConstantTok}[1]{\textcolor[rgb]{0.53,0.00,0.00}{{#1}}}
    \newcommand{\SpecialCharTok}[1]{\textcolor[rgb]{0.25,0.44,0.63}{{#1}}}
    \newcommand{\VerbatimStringTok}[1]{\textcolor[rgb]{0.25,0.44,0.63}{{#1}}}
    \newcommand{\SpecialStringTok}[1]{\textcolor[rgb]{0.73,0.40,0.53}{{#1}}}
    \newcommand{\ImportTok}[1]{{#1}}
    \newcommand{\DocumentationTok}[1]{\textcolor[rgb]{0.73,0.13,0.13}{\textit{{#1}}}}
    \newcommand{\AnnotationTok}[1]{\textcolor[rgb]{0.38,0.63,0.69}{\textbf{\textit{{#1}}}}}
    \newcommand{\CommentVarTok}[1]{\textcolor[rgb]{0.38,0.63,0.69}{\textbf{\textit{{#1}}}}}
    \newcommand{\VariableTok}[1]{\textcolor[rgb]{0.10,0.09,0.49}{{#1}}}
    \newcommand{\ControlFlowTok}[1]{\textcolor[rgb]{0.00,0.44,0.13}{\textbf{{#1}}}}
    \newcommand{\OperatorTok}[1]{\textcolor[rgb]{0.40,0.40,0.40}{{#1}}}
    \newcommand{\BuiltInTok}[1]{{#1}}
    \newcommand{\ExtensionTok}[1]{{#1}}
    \newcommand{\PreprocessorTok}[1]{\textcolor[rgb]{0.74,0.48,0.00}{{#1}}}
    \newcommand{\AttributeTok}[1]{\textcolor[rgb]{0.49,0.56,0.16}{{#1}}}
    \newcommand{\InformationTok}[1]{\textcolor[rgb]{0.38,0.63,0.69}{\textbf{\textit{{#1}}}}}
    \newcommand{\WarningTok}[1]{\textcolor[rgb]{0.38,0.63,0.69}{\textbf{\textit{{#1}}}}}
    
    
    % Define a nice break command that doesn't care if a line doesn't already
    % exist.
    \def\br{\hspace*{\fill} \\* }
    % Math Jax compatability definitions
    \def\gt{>}
    \def\lt{<}
    % Document parameters
    \title{parkinsons\_disease\_replicated\_acoustic\_features}
    
    
    

    % Pygments definitions
    
\makeatletter
\def\PY@reset{\let\PY@it=\relax \let\PY@bf=\relax%
    \let\PY@ul=\relax \let\PY@tc=\relax%
    \let\PY@bc=\relax \let\PY@ff=\relax}
\def\PY@tok#1{\csname PY@tok@#1\endcsname}
\def\PY@toks#1+{\ifx\relax#1\empty\else%
    \PY@tok{#1}\expandafter\PY@toks\fi}
\def\PY@do#1{\PY@bc{\PY@tc{\PY@ul{%
    \PY@it{\PY@bf{\PY@ff{#1}}}}}}}
\def\PY#1#2{\PY@reset\PY@toks#1+\relax+\PY@do{#2}}

\expandafter\def\csname PY@tok@w\endcsname{\def\PY@tc##1{\textcolor[rgb]{0.73,0.73,0.73}{##1}}}
\expandafter\def\csname PY@tok@c\endcsname{\let\PY@it=\textit\def\PY@tc##1{\textcolor[rgb]{0.25,0.50,0.50}{##1}}}
\expandafter\def\csname PY@tok@cp\endcsname{\def\PY@tc##1{\textcolor[rgb]{0.74,0.48,0.00}{##1}}}
\expandafter\def\csname PY@tok@k\endcsname{\let\PY@bf=\textbf\def\PY@tc##1{\textcolor[rgb]{0.00,0.50,0.00}{##1}}}
\expandafter\def\csname PY@tok@kp\endcsname{\def\PY@tc##1{\textcolor[rgb]{0.00,0.50,0.00}{##1}}}
\expandafter\def\csname PY@tok@kt\endcsname{\def\PY@tc##1{\textcolor[rgb]{0.69,0.00,0.25}{##1}}}
\expandafter\def\csname PY@tok@o\endcsname{\def\PY@tc##1{\textcolor[rgb]{0.40,0.40,0.40}{##1}}}
\expandafter\def\csname PY@tok@ow\endcsname{\let\PY@bf=\textbf\def\PY@tc##1{\textcolor[rgb]{0.67,0.13,1.00}{##1}}}
\expandafter\def\csname PY@tok@nb\endcsname{\def\PY@tc##1{\textcolor[rgb]{0.00,0.50,0.00}{##1}}}
\expandafter\def\csname PY@tok@nf\endcsname{\def\PY@tc##1{\textcolor[rgb]{0.00,0.00,1.00}{##1}}}
\expandafter\def\csname PY@tok@nc\endcsname{\let\PY@bf=\textbf\def\PY@tc##1{\textcolor[rgb]{0.00,0.00,1.00}{##1}}}
\expandafter\def\csname PY@tok@nn\endcsname{\let\PY@bf=\textbf\def\PY@tc##1{\textcolor[rgb]{0.00,0.00,1.00}{##1}}}
\expandafter\def\csname PY@tok@ne\endcsname{\let\PY@bf=\textbf\def\PY@tc##1{\textcolor[rgb]{0.82,0.25,0.23}{##1}}}
\expandafter\def\csname PY@tok@nv\endcsname{\def\PY@tc##1{\textcolor[rgb]{0.10,0.09,0.49}{##1}}}
\expandafter\def\csname PY@tok@no\endcsname{\def\PY@tc##1{\textcolor[rgb]{0.53,0.00,0.00}{##1}}}
\expandafter\def\csname PY@tok@nl\endcsname{\def\PY@tc##1{\textcolor[rgb]{0.63,0.63,0.00}{##1}}}
\expandafter\def\csname PY@tok@ni\endcsname{\let\PY@bf=\textbf\def\PY@tc##1{\textcolor[rgb]{0.60,0.60,0.60}{##1}}}
\expandafter\def\csname PY@tok@na\endcsname{\def\PY@tc##1{\textcolor[rgb]{0.49,0.56,0.16}{##1}}}
\expandafter\def\csname PY@tok@nt\endcsname{\let\PY@bf=\textbf\def\PY@tc##1{\textcolor[rgb]{0.00,0.50,0.00}{##1}}}
\expandafter\def\csname PY@tok@nd\endcsname{\def\PY@tc##1{\textcolor[rgb]{0.67,0.13,1.00}{##1}}}
\expandafter\def\csname PY@tok@s\endcsname{\def\PY@tc##1{\textcolor[rgb]{0.73,0.13,0.13}{##1}}}
\expandafter\def\csname PY@tok@sd\endcsname{\let\PY@it=\textit\def\PY@tc##1{\textcolor[rgb]{0.73,0.13,0.13}{##1}}}
\expandafter\def\csname PY@tok@si\endcsname{\let\PY@bf=\textbf\def\PY@tc##1{\textcolor[rgb]{0.73,0.40,0.53}{##1}}}
\expandafter\def\csname PY@tok@se\endcsname{\let\PY@bf=\textbf\def\PY@tc##1{\textcolor[rgb]{0.73,0.40,0.13}{##1}}}
\expandafter\def\csname PY@tok@sr\endcsname{\def\PY@tc##1{\textcolor[rgb]{0.73,0.40,0.53}{##1}}}
\expandafter\def\csname PY@tok@ss\endcsname{\def\PY@tc##1{\textcolor[rgb]{0.10,0.09,0.49}{##1}}}
\expandafter\def\csname PY@tok@sx\endcsname{\def\PY@tc##1{\textcolor[rgb]{0.00,0.50,0.00}{##1}}}
\expandafter\def\csname PY@tok@m\endcsname{\def\PY@tc##1{\textcolor[rgb]{0.40,0.40,0.40}{##1}}}
\expandafter\def\csname PY@tok@gh\endcsname{\let\PY@bf=\textbf\def\PY@tc##1{\textcolor[rgb]{0.00,0.00,0.50}{##1}}}
\expandafter\def\csname PY@tok@gu\endcsname{\let\PY@bf=\textbf\def\PY@tc##1{\textcolor[rgb]{0.50,0.00,0.50}{##1}}}
\expandafter\def\csname PY@tok@gd\endcsname{\def\PY@tc##1{\textcolor[rgb]{0.63,0.00,0.00}{##1}}}
\expandafter\def\csname PY@tok@gi\endcsname{\def\PY@tc##1{\textcolor[rgb]{0.00,0.63,0.00}{##1}}}
\expandafter\def\csname PY@tok@gr\endcsname{\def\PY@tc##1{\textcolor[rgb]{1.00,0.00,0.00}{##1}}}
\expandafter\def\csname PY@tok@ge\endcsname{\let\PY@it=\textit}
\expandafter\def\csname PY@tok@gs\endcsname{\let\PY@bf=\textbf}
\expandafter\def\csname PY@tok@gp\endcsname{\let\PY@bf=\textbf\def\PY@tc##1{\textcolor[rgb]{0.00,0.00,0.50}{##1}}}
\expandafter\def\csname PY@tok@go\endcsname{\def\PY@tc##1{\textcolor[rgb]{0.53,0.53,0.53}{##1}}}
\expandafter\def\csname PY@tok@gt\endcsname{\def\PY@tc##1{\textcolor[rgb]{0.00,0.27,0.87}{##1}}}
\expandafter\def\csname PY@tok@err\endcsname{\def\PY@bc##1{\setlength{\fboxsep}{0pt}\fcolorbox[rgb]{1.00,0.00,0.00}{1,1,1}{\strut ##1}}}
\expandafter\def\csname PY@tok@kc\endcsname{\let\PY@bf=\textbf\def\PY@tc##1{\textcolor[rgb]{0.00,0.50,0.00}{##1}}}
\expandafter\def\csname PY@tok@kd\endcsname{\let\PY@bf=\textbf\def\PY@tc##1{\textcolor[rgb]{0.00,0.50,0.00}{##1}}}
\expandafter\def\csname PY@tok@kn\endcsname{\let\PY@bf=\textbf\def\PY@tc##1{\textcolor[rgb]{0.00,0.50,0.00}{##1}}}
\expandafter\def\csname PY@tok@kr\endcsname{\let\PY@bf=\textbf\def\PY@tc##1{\textcolor[rgb]{0.00,0.50,0.00}{##1}}}
\expandafter\def\csname PY@tok@bp\endcsname{\def\PY@tc##1{\textcolor[rgb]{0.00,0.50,0.00}{##1}}}
\expandafter\def\csname PY@tok@fm\endcsname{\def\PY@tc##1{\textcolor[rgb]{0.00,0.00,1.00}{##1}}}
\expandafter\def\csname PY@tok@vc\endcsname{\def\PY@tc##1{\textcolor[rgb]{0.10,0.09,0.49}{##1}}}
\expandafter\def\csname PY@tok@vg\endcsname{\def\PY@tc##1{\textcolor[rgb]{0.10,0.09,0.49}{##1}}}
\expandafter\def\csname PY@tok@vi\endcsname{\def\PY@tc##1{\textcolor[rgb]{0.10,0.09,0.49}{##1}}}
\expandafter\def\csname PY@tok@vm\endcsname{\def\PY@tc##1{\textcolor[rgb]{0.10,0.09,0.49}{##1}}}
\expandafter\def\csname PY@tok@sa\endcsname{\def\PY@tc##1{\textcolor[rgb]{0.73,0.13,0.13}{##1}}}
\expandafter\def\csname PY@tok@sb\endcsname{\def\PY@tc##1{\textcolor[rgb]{0.73,0.13,0.13}{##1}}}
\expandafter\def\csname PY@tok@sc\endcsname{\def\PY@tc##1{\textcolor[rgb]{0.73,0.13,0.13}{##1}}}
\expandafter\def\csname PY@tok@dl\endcsname{\def\PY@tc##1{\textcolor[rgb]{0.73,0.13,0.13}{##1}}}
\expandafter\def\csname PY@tok@s2\endcsname{\def\PY@tc##1{\textcolor[rgb]{0.73,0.13,0.13}{##1}}}
\expandafter\def\csname PY@tok@sh\endcsname{\def\PY@tc##1{\textcolor[rgb]{0.73,0.13,0.13}{##1}}}
\expandafter\def\csname PY@tok@s1\endcsname{\def\PY@tc##1{\textcolor[rgb]{0.73,0.13,0.13}{##1}}}
\expandafter\def\csname PY@tok@mb\endcsname{\def\PY@tc##1{\textcolor[rgb]{0.40,0.40,0.40}{##1}}}
\expandafter\def\csname PY@tok@mf\endcsname{\def\PY@tc##1{\textcolor[rgb]{0.40,0.40,0.40}{##1}}}
\expandafter\def\csname PY@tok@mh\endcsname{\def\PY@tc##1{\textcolor[rgb]{0.40,0.40,0.40}{##1}}}
\expandafter\def\csname PY@tok@mi\endcsname{\def\PY@tc##1{\textcolor[rgb]{0.40,0.40,0.40}{##1}}}
\expandafter\def\csname PY@tok@il\endcsname{\def\PY@tc##1{\textcolor[rgb]{0.40,0.40,0.40}{##1}}}
\expandafter\def\csname PY@tok@mo\endcsname{\def\PY@tc##1{\textcolor[rgb]{0.40,0.40,0.40}{##1}}}
\expandafter\def\csname PY@tok@ch\endcsname{\let\PY@it=\textit\def\PY@tc##1{\textcolor[rgb]{0.25,0.50,0.50}{##1}}}
\expandafter\def\csname PY@tok@cm\endcsname{\let\PY@it=\textit\def\PY@tc##1{\textcolor[rgb]{0.25,0.50,0.50}{##1}}}
\expandafter\def\csname PY@tok@cpf\endcsname{\let\PY@it=\textit\def\PY@tc##1{\textcolor[rgb]{0.25,0.50,0.50}{##1}}}
\expandafter\def\csname PY@tok@c1\endcsname{\let\PY@it=\textit\def\PY@tc##1{\textcolor[rgb]{0.25,0.50,0.50}{##1}}}
\expandafter\def\csname PY@tok@cs\endcsname{\let\PY@it=\textit\def\PY@tc##1{\textcolor[rgb]{0.25,0.50,0.50}{##1}}}

\def\PYZbs{\char`\\}
\def\PYZus{\char`\_}
\def\PYZob{\char`\{}
\def\PYZcb{\char`\}}
\def\PYZca{\char`\^}
\def\PYZam{\char`\&}
\def\PYZlt{\char`\<}
\def\PYZgt{\char`\>}
\def\PYZsh{\char`\#}
\def\PYZpc{\char`\%}
\def\PYZdl{\char`\$}
\def\PYZhy{\char`\-}
\def\PYZsq{\char`\'}
\def\PYZdq{\char`\"}
\def\PYZti{\char`\~}
% for compatibility with earlier versions
\def\PYZat{@}
\def\PYZlb{[}
\def\PYZrb{]}
\makeatother


    % Exact colors from NB
    \definecolor{incolor}{rgb}{0.0, 0.0, 0.5}
    \definecolor{outcolor}{rgb}{0.545, 0.0, 0.0}



    
    % Prevent overflowing lines due to hard-to-break entities
    \sloppy 
    % Setup hyperref package
    \hypersetup{
      breaklinks=true,  % so long urls are correctly broken across lines
      colorlinks=true,
      urlcolor=urlcolor,
      linkcolor=linkcolor,
      citecolor=citecolor,
      }
    % Slightly bigger margins than the latex defaults
    
    \geometry{verbose,tmargin=1in,bmargin=1in,lmargin=1in,rmargin=1in}
    
    

    \begin{document}
    
    
    \maketitle
    
    

    
    \hypertarget{parkinsons-disease-classification-with-replicated-acoustic-features}{%
\section{Parkinson's Disease Classification with Replicated Acoustic
Features}\label{parkinsons-disease-classification-with-replicated-acoustic-features}}

\begin{itemize}
\item
  The Dataset is a collection of acoustic features extracted from 3
  samples of voice from 80 people of which 40 were tested positive for
  Parkinson's Disease.
\item
\begin{verbatim}
80 persons
3 voice samples per person
80 * 3 = 240 samples in total
\end{verbatim}
\end{itemize}

\hypertarget{abstract}{%
\subsection{\texorpdfstring{\textbf{Abstract}}{Abstract}}\label{abstract}}

\begin{itemize}
\tightlist
\item
  Contains acoustic features extracted from 3 voice recording
  replications of the sustained /a/ phonation for each one of the 80
  subjects (40 tested positive).
\end{itemize}

\hypertarget{dataset-information}{%
\subsection{\texorpdfstring{\textbf{Dataset
Information}}{Dataset Information}}\label{dataset-information}}

\begin{itemize}
\tightlist
\item
  \textbf{Data Set Characteristics}: Multivariate
\item
  \textbf{Number of Instances}: 240
\item
  \textbf{Area}: Life
\item
  \textbf{Number of Attributes}: 46
\item
  \textbf{Associated Tasks}: Classification
\item
  \textbf{Date Donated}: 10th April 2019
\end{itemize}

\hypertarget{dataset-links}{%
\subsection{\texorpdfstring{\textbf{Dataset
Links}}{Dataset Links}}\label{dataset-links}}

\begin{itemize}
\tightlist
\item
  \href{https://archive.ics.uci.edu/ml/machine-learning-databases/00489/ReplicatedAcousticFeatures-ParkinsonDatabase.csv}{dataset-csv}
\item
  \href{https://archive.ics.uci.edu/ml/datasets/Parkinson+Dataset+with+replicated+acoustic+features+}{dataset-description}
\end{itemize}

\hypertarget{information}{%
\subsection{\texorpdfstring{\textbf{Information}}{Information}}\label{information}}

\begin{enumerate}
\def\labelenumi{\arabic{enumi}.}
\tightlist
\item
  Each row can not be used independently, because is one of the three
  replications of one individual. Nature of data is dependent for each
  subject, but independent from one to another subject. So, traditional
  technique from machine learning can not be applied to this dataset,
  because those techniques are based on the independent nature of the
  instances. There are 240 instances but for only 80 subjects, so they
  are not independent. Techniques as those presented in Naranjo et
  al.~(2016), Naranjo et al.~(2017) or other specifically designed can
  be used.
\item
  The concept of replication considered here does not match the
  classical concept of statistical repeated measurements. The term
  `replications' refers to the collection of features extracted from
  voice recordings belonging to the same subject. Since, in this
  context, features are extracted from multiple consecutive voice
  recordings from the same subject, in principle, the features should be
  identical. The imperfections in technology and the own biological
  variability result in non-identical replicated features that are more
  similar to one another than features from different subjects.
\item
  All information about how the dataset was generated is presented in
  Naranjo et al.~(2016).
\end{enumerate}

\hypertarget{attribute-information}{%
\subsection{\texorpdfstring{\textbf{Attribute
Information}}{Attribute Information}}\label{attribute-information}}

\begin{enumerate}
\def\labelenumi{\arabic{enumi}.}
\tightlist
\item
  \textbf{ID}: Subjects's identifier.
\item
  \textbf{Recording}: Number of the recording.
\item
  \textbf{Status}: 0=Healthy; 1=PD
\item
  \textbf{Gender}: 0=Man; 1=Woman
\item
  Pitch local perturbation measures: relative jitter
  (\textbf{Jitter\_rel}), absolute jitter (\textbf{Jitter\_abs}),
  relative average perturbation (\textbf{Jitter\_RAP}), and pitch
  perturbation quotient (\textbf{Jitter\_PPQ}).
\item
  Amplitude perturbation measures: local shimmer (\textbf{Shim\_loc}),
  shimmer in dB (\textbf{Shim\_dB}), 3-point amplitude perturbation
  quotient (\textbf{Shim\_APQ3}), 5-point amplitude perturbation
  quotient (\textbf{Shim\_APQ5}), and 11-point amplitude perturbation
  quotient (\textbf{Shim\_APQ11}).
\item
  Harmonic-to-noise ratio measures: harmonic-to-noise ratio in the
  frequency band 0-500 Hz (\textbf{HNR05}), in 0-1500 Hz
  (\textbf{HNR15}), in 0-2500 Hz (\textbf{HNR25}), in 0-3500 Hz
  (\textbf{HNR35}), and in 0-3800 Hz (\textbf{HNR38}).
\item
  Mel frequency cepstral coefficient-based spectral measures of order 0
  to 12 (\textbf{MFCC0, MFCC1,\ldots, MFCC12}) and their derivatives
  (\textbf{Delta0, Delta1,\ldots, Delta12}).
\item
  Recurrence period density entropy (\textbf{RPDE}).
\item
  Detrended fluctuation analysis (\textbf{DFA}).
\item
  Pitch period entropy (\textbf{PPE}).
\item
  Glottal-to-noise excitation ratio (\textbf{GNE}).
\end{enumerate}

    \begin{Verbatim}[commandchars=\\\{\}]
{\color{incolor}In [{\color{incolor}1}]:} \PY{c+c1}{\PYZsh{}@title Getting the Dataset from the [dataset\PYZhy{}link](https://archive.ics.uci.edu/ml/machine\PYZhy{}learning\PYZhy{}databases/00489/ReplicatedAcousticFeatures\PYZhy{}ParkinsonDatabase.csv).}
        \PY{o}{!}wget \PYZhy{}q https://archive.ics.uci.edu/ml/machine\PYZhy{}learning\PYZhy{}databases/00489/ReplicatedAcousticFeatures\PYZhy{}ParkinsonDatabase.csv
\end{Verbatim}


    \begin{Verbatim}[commandchars=\\\{\}]
{\color{incolor}In [{\color{incolor}2}]:} \PY{c+c1}{\PYZsh{}@title Importing the Necessary Libraries.}
        \PY{k+kn}{import} \PY{n+nn}{pandas} \PY{k}{as} \PY{n+nn}{pd}
        \PY{k+kn}{import} \PY{n+nn}{matplotlib}\PY{n+nn}{.}\PY{n+nn}{pyplot} \PY{k}{as} \PY{n+nn}{plt}
        \PY{k+kn}{import} \PY{n+nn}{seaborn} \PY{k}{as} \PY{n+nn}{sns}
        
        \PY{k+kn}{from} \PY{n+nn}{sklearn}\PY{n+nn}{.}\PY{n+nn}{decomposition} \PY{k}{import} \PY{n}{PCA}
        \PY{k+kn}{from} \PY{n+nn}{sklearn}\PY{n+nn}{.}\PY{n+nn}{preprocessing} \PY{k}{import} \PY{n}{StandardScaler}
        
        \PY{k+kn}{from} \PY{n+nn}{sklearn}\PY{n+nn}{.}\PY{n+nn}{model\PYZus{}selection} \PY{k}{import} \PY{n}{train\PYZus{}test\PYZus{}split}
        
        \PY{k+kn}{from} \PY{n+nn}{sklearn}\PY{n+nn}{.}\PY{n+nn}{ensemble} \PY{k}{import} \PY{n}{RandomForestRegressor}
        
        \PY{k+kn}{from} \PY{n+nn}{sklearn}\PY{n+nn}{.}\PY{n+nn}{metrics} \PY{k}{import} \PY{n}{confusion\PYZus{}matrix}
        \PY{k+kn}{from} \PY{n+nn}{sklearn}\PY{n+nn}{.}\PY{n+nn}{metrics} \PY{k}{import} \PY{n}{f1\PYZus{}score}
        \PY{k+kn}{from} \PY{n+nn}{sklearn}\PY{n+nn}{.}\PY{n+nn}{metrics} \PY{k}{import} \PY{n}{accuracy\PYZus{}score}
        \PY{k+kn}{import} \PY{n+nn}{numpy} \PY{k}{as} \PY{n+nn}{np}
\end{Verbatim}


    \begin{Verbatim}[commandchars=\\\{\}]
{\color{incolor}In [{\color{incolor}3}]:} \PY{c+c1}{\PYZsh{}@title Converting the csvdata to a [pandas.DataFrame](https://pandas.pydata.org/pandas\PYZhy{}docs/stable/reference/api/pandas.DataFrame.html).}
        \PY{n}{dataset} \PY{o}{=} \PY{n}{pd}\PY{o}{.}\PY{n}{read\PYZus{}csv}\PY{p}{(}\PY{l+s+s2}{\PYZdq{}}\PY{l+s+s2}{ReplicatedAcousticFeatures\PYZhy{}ParkinsonDatabase.csv}\PY{l+s+s2}{\PYZdq{}}\PY{p}{)}
        \PY{n}{dataset} \PY{o}{=} \PY{n}{dataset}\PY{o}{.}\PY{n}{drop}\PY{p}{(}\PY{l+s+s1}{\PYZsq{}}\PY{l+s+s1}{ID}\PY{l+s+s1}{\PYZsq{}}\PY{p}{,}\PY{n}{axis}\PY{o}{=}\PY{l+m+mi}{1}\PY{p}{)}
        \PY{n}{x} \PY{o}{=} \PY{n}{dataset}\PY{o}{.}\PY{n}{drop}\PY{p}{(}\PY{l+s+s1}{\PYZsq{}}\PY{l+s+s1}{Status}\PY{l+s+s1}{\PYZsq{}}\PY{p}{,}\PY{n}{axis}\PY{o}{=}\PY{l+m+mi}{1}\PY{p}{,}\PY{n}{inplace}\PY{o}{=}\PY{k+kc}{False}\PY{p}{)}
        \PY{n}{y} \PY{o}{=} \PY{n}{dataset}\PY{p}{[}\PY{l+s+s1}{\PYZsq{}}\PY{l+s+s1}{Status}\PY{l+s+s1}{\PYZsq{}}\PY{p}{]}
        \PY{n}{dataset}\PY{o}{.}\PY{n}{head}\PY{p}{(}\PY{p}{)}
\end{Verbatim}


\begin{Verbatim}[commandchars=\\\{\}]
{\color{outcolor}Out[{\color{outcolor}3}]:}    Recording  Status  Gender  Jitter\_rel  Jitter\_abs  Jitter\_RAP  Jitter\_PPQ  \textbackslash{}
        0          1       0       1     0.25546    0.000015    0.001467    0.001673   
        1          2       0       1     0.36964    0.000022    0.001932    0.002245   
        2          3       0       1     0.23514    0.000013    0.001353    0.001546   
        3          1       0       0     0.29320    0.000017    0.001105    0.001444   
        4          2       0       0     0.23075    0.000015    0.001073    0.001404   
        
           Shim\_loc  Shim\_dB  Shim\_APQ3  {\ldots}    Delta3    Delta4    Delta5    Delta6  \textbackslash{}
        0  0.030256  0.26313   0.017463  {\ldots}  1.407701  1.417218  1.380352  1.420670   
        1  0.023146  0.20217   0.013010  {\ldots}  1.331232  1.227338  1.213377  1.352739   
        2  0.019338  0.16710   0.011049  {\ldots}  1.412304  1.324674  1.276088  1.429634   
        3  0.024716  0.20892   0.014525  {\ldots}  1.501200  1.534170  1.323993  1.496442   
        4  0.013119  0.11607   0.006461  {\ldots}  1.508468  1.334511  1.610694  1.685021   
        
             Delta7    Delta8    Delta9   Delta10   Delta11   Delta12  
        0  1.451240  1.440295  1.403678  1.405495  1.416705  1.354610  
        1  1.354242  1.365692  1.322870  1.314549  1.318999  1.323508  
        2  1.455996  1.368882  1.438053  1.388910  1.305469  1.305402  
        3  1.472926  1.643177  1.551286  1.638346  1.604008  1.621456  
        4  1.417614  1.574895  1.640088  1.533666  1.297536  1.382023  
        
        [5 rows x 47 columns]
\end{Verbatim}
            
    \begin{Verbatim}[commandchars=\\\{\}]
{\color{incolor}In [{\color{incolor}4}]:} \PY{c+c1}{\PYZsh{}@title Plotting a Correlation HeatMap for the Attributes in the Dataset.}
        \PY{o}{\PYZpc{}}\PY{k}{matplotlib} inline
        \PY{n}{corrMatrix} \PY{o}{=} \PY{n}{dataset}\PY{o}{.}\PY{n}{corr}\PY{p}{(}\PY{p}{)}
        \PY{n}{fig}\PY{p}{,} \PY{n}{ax} \PY{o}{=} \PY{n}{plt}\PY{o}{.}\PY{n}{subplots}\PY{p}{(}\PY{n}{figsize}\PY{o}{=}\PY{p}{(}\PY{l+m+mi}{10}\PY{p}{,}\PY{l+m+mi}{10}\PY{p}{)}\PY{p}{)}
        \PY{n}{hm} \PY{o}{=} \PY{n}{sns}\PY{o}{.}\PY{n}{heatmap}\PY{p}{(}\PY{n}{dataset}\PY{o}{.}\PY{n}{corr}\PY{p}{(}\PY{p}{)}\PY{p}{,}\PY{n}{xticklabels}\PY{o}{=}\PY{k+kc}{True}\PY{p}{,} \PY{n}{yticklabels}\PY{o}{=}\PY{k+kc}{True}\PY{p}{,}\PY{n}{linewidths}\PY{o}{=}\PY{l+m+mi}{1}\PY{p}{)}
        \PY{n}{hm}\PY{o}{.}\PY{n}{set\PYZus{}title}\PY{p}{(}\PY{l+s+s2}{\PYZdq{}}\PY{l+s+s2}{Heat Map}\PY{l+s+s2}{\PYZdq{}}\PY{p}{)}
\end{Verbatim}


\begin{Verbatim}[commandchars=\\\{\}]
{\color{outcolor}Out[{\color{outcolor}4}]:} Text(0.5, 1, 'Heat Map')
\end{Verbatim}
            
    \begin{center}
    \adjustimage{max size={0.9\linewidth}{0.9\paperheight}}{output_4_1.png}
    \end{center}
    { \hspace*{\fill} \\}
    
    \begin{Verbatim}[commandchars=\\\{\}]
{\color{incolor}In [{\color{incolor}5}]:} \PY{c+c1}{\PYZsh{}@title Performing Principal Component Analysis for the given Dataset.}
        \PY{n}{scaler}\PY{o}{=}\PY{n}{StandardScaler}\PY{p}{(}\PY{p}{)}
        \PY{n}{x} \PY{o}{=} \PY{n}{scaler}\PY{o}{.}\PY{n}{fit\PYZus{}transform}\PY{p}{(}\PY{n}{x}\PY{p}{)}
        
        \PY{n}{components} \PY{o}{=} \PY{l+m+mi}{10}
        \PY{n}{pca} \PY{o}{=} \PY{n}{PCA}\PY{p}{(}\PY{n}{n\PYZus{}components}\PY{o}{=}\PY{n}{components}\PY{p}{)}
        \PY{n}{principalComponents} \PY{o}{=} \PY{n}{pca}\PY{o}{.}\PY{n}{fit\PYZus{}transform}\PY{p}{(}\PY{n}{x}\PY{p}{)}
        \PY{n}{x} \PY{o}{=} \PY{n}{pd}\PY{o}{.}\PY{n}{DataFrame}\PY{p}{(}\PY{n}{data} \PY{o}{=} \PY{n}{principalComponents}\PY{p}{,} \PY{n}{columns} \PY{o}{=} \PY{p}{[}\PY{l+s+s2}{\PYZdq{}}\PY{l+s+s2}{pc}\PY{l+s+s2}{\PYZdq{}}\PY{o}{+}\PY{n+nb}{str}\PY{p}{(}\PY{n}{i}\PY{p}{)} \PY{k}{for} \PY{n}{i} \PY{o+ow}{in} \PY{n+nb}{range}\PY{p}{(}\PY{n}{components}\PY{p}{)}\PY{p}{]}\PY{p}{)}
        
        \PY{n}{ax} \PY{o}{=} \PY{n}{sns}\PY{o}{.}\PY{n}{barplot}\PY{p}{(} \PY{n}{x}\PY{o}{=}\PY{p}{[}\PY{l+s+s2}{\PYZdq{}}\PY{l+s+s2}{pc}\PY{l+s+s2}{\PYZdq{}}\PY{o}{+}\PY{n+nb}{str}\PY{p}{(}\PY{n}{i}\PY{p}{)} \PY{k}{for} \PY{n}{i} \PY{o+ow}{in} \PY{n+nb}{range}\PY{p}{(}\PY{n}{components}\PY{p}{)}\PY{p}{]}\PY{p}{,}\PY{n}{y}\PY{o}{=}\PY{n}{pca}\PY{o}{.}\PY{n}{explained\PYZus{}variance\PYZus{}}\PY{p}{)}
        \PY{n}{ax}\PY{o}{.}\PY{n}{set\PYZus{}title}\PY{p}{(}\PY{l+s+s2}{\PYZdq{}}\PY{l+s+s2}{PCA}\PY{l+s+s2}{\PYZdq{}}\PY{p}{)}
\end{Verbatim}


    \begin{Verbatim}[commandchars=\\\{\}]
/usr/local/lib/python3.7/site-packages/sklearn/preprocessing/data.py:645: DataConversionWarning: Data with input dtype int64, float64 were all converted to float64 by StandardScaler.
  return self.partial\_fit(X, y)
/usr/local/lib/python3.7/site-packages/sklearn/base.py:464: DataConversionWarning: Data with input dtype int64, float64 were all converted to float64 by StandardScaler.
  return self.fit(X, **fit\_params).transform(X)

    \end{Verbatim}

\begin{Verbatim}[commandchars=\\\{\}]
{\color{outcolor}Out[{\color{outcolor}5}]:} Text(0.5, 1.0, 'PCA')
\end{Verbatim}
            
    \begin{center}
    \adjustimage{max size={0.9\linewidth}{0.9\paperheight}}{output_5_2.png}
    \end{center}
    { \hspace*{\fill} \\}
    
    \begin{Verbatim}[commandchars=\\\{\}]
{\color{incolor}In [{\color{incolor}6}]:} \PY{c+c1}{\PYZsh{}@title Performing Train Test Split with 80\PYZhy{}20.}
        \PY{n}{x\PYZus{}train}\PY{p}{,} \PY{n}{x\PYZus{}test}\PY{p}{,} \PY{n}{y\PYZus{}train}\PY{p}{,} \PY{n}{y\PYZus{}test} \PY{o}{=} \PY{n}{train\PYZus{}test\PYZus{}split}\PY{p}{(}\PY{n}{x}\PY{p}{,} \PY{n}{y}\PY{p}{,} \PY{n}{test\PYZus{}size}\PY{o}{=}\PY{l+m+mf}{0.2}\PY{p}{,} \PY{n}{random\PYZus{}state}\PY{o}{=}\PY{l+m+mi}{30}\PY{p}{)}
\end{Verbatim}


    \begin{Verbatim}[commandchars=\\\{\}]
{\color{incolor}In [{\color{incolor}7}]:} \PY{c+c1}{\PYZsh{}@title Choosing the Optimal Random State}
        \PY{n}{accuracy\PYZus{}max} \PY{o}{=} \PY{l+m+mi}{0}
        \PY{n}{random\PYZus{}state\PYZus{}choice} \PY{o}{=} \PY{l+m+mi}{0}
        \PY{k}{for} \PY{n}{state} \PY{o+ow}{in} \PY{n+nb}{range}\PY{p}{(}\PY{l+m+mi}{0}\PY{p}{,} \PY{l+m+mi}{50}\PY{p}{)}\PY{p}{:}
          \PY{n}{rf} \PY{o}{=} \PY{n}{RandomForestRegressor}\PY{p}{(}\PY{n}{n\PYZus{}estimators} \PY{o}{=} \PY{l+m+mi}{2500}\PY{p}{,} \PY{n}{random\PYZus{}state} \PY{o}{=} \PY{n}{state}\PY{p}{)}
          \PY{n}{rf}\PY{o}{.}\PY{n}{fit}\PY{p}{(}\PY{n}{x\PYZus{}train}\PY{p}{,} \PY{n}{y\PYZus{}train}\PY{p}{)}
          \PY{n}{y\PYZus{}pred} \PY{o}{=} \PY{n}{np}\PY{o}{.}\PY{n}{round}\PY{p}{(}\PY{n}{rf}\PY{o}{.}\PY{n}{predict}\PY{p}{(}\PY{n}{x\PYZus{}test}\PY{p}{)}\PY{p}{)}
          \PY{n}{accuracy} \PY{o}{=} \PY{n}{accuracy\PYZus{}score}\PY{p}{(}\PY{n}{y\PYZus{}test}\PY{p}{,} \PY{n}{y\PYZus{}pred}\PY{p}{)}
          \PY{k}{if} \PY{n}{accuracy} \PY{o}{\PYZgt{}} \PY{n}{accuracy\PYZus{}max}\PY{p}{:}
            \PY{n}{random\PYZus{}state\PYZus{}choice} \PY{o}{=} \PY{n}{state}
        
        \PY{n+nb}{print}\PY{p}{(}\PY{l+s+s1}{\PYZsq{}}\PY{l+s+s1}{random\PYZus{}state\PYZus{}choice: }\PY{l+s+s1}{\PYZsq{}} \PY{o}{+} \PY{n+nb}{str}\PY{p}{(}\PY{n}{random\PYZus{}state\PYZus{}choice}\PY{p}{)}\PY{p}{)}
\end{Verbatim}


    \begin{Verbatim}[commandchars=\\\{\}]
random\_state\_choice: 49

    \end{Verbatim}

    \begin{Verbatim}[commandchars=\\\{\}]
{\color{incolor}In [{\color{incolor}8}]:} \PY{c+c1}{\PYZsh{}@title Choosing the Optimal Number of Estimators}
        \PY{n}{accuracy\PYZus{}max} \PY{o}{=} \PY{l+m+mi}{0}
        \PY{n}{estimator\PYZus{}choice} \PY{o}{=} \PY{l+m+mi}{0}
        \PY{k}{for} \PY{n}{estimator} \PY{o+ow}{in} \PY{n+nb}{range}\PY{p}{(}\PY{l+m+mi}{500}\PY{p}{,} \PY{l+m+mi}{5000}\PY{p}{,} \PY{l+m+mi}{500}\PY{p}{)}\PY{p}{:}
          \PY{n}{rf} \PY{o}{=} \PY{n}{RandomForestRegressor}\PY{p}{(}\PY{n}{n\PYZus{}estimators} \PY{o}{=} \PY{n}{estimator}\PY{p}{,} \PY{n}{random\PYZus{}state} \PY{o}{=} \PY{n}{random\PYZus{}state\PYZus{}choice}\PY{p}{)}
          \PY{n}{rf}\PY{o}{.}\PY{n}{fit}\PY{p}{(}\PY{n}{x\PYZus{}train}\PY{p}{,} \PY{n}{y\PYZus{}train}\PY{p}{)}
          \PY{n}{y\PYZus{}pred} \PY{o}{=} \PY{n}{np}\PY{o}{.}\PY{n}{round}\PY{p}{(}\PY{n}{rf}\PY{o}{.}\PY{n}{predict}\PY{p}{(}\PY{n}{x\PYZus{}test}\PY{p}{)}\PY{p}{)}
          \PY{n}{accuracy} \PY{o}{=} \PY{n}{accuracy\PYZus{}score}\PY{p}{(}\PY{n}{y\PYZus{}test}\PY{p}{,} \PY{n}{y\PYZus{}pred}\PY{p}{)}
          \PY{k}{if} \PY{n}{accuracy} \PY{o}{\PYZgt{}} \PY{n}{accuracy\PYZus{}max}\PY{p}{:}
            \PY{n}{estimator\PYZus{}choice} \PY{o}{=} \PY{n}{estimator}
        
        \PY{n+nb}{print}\PY{p}{(}\PY{l+s+s1}{\PYZsq{}}\PY{l+s+s1}{estimator\PYZus{}choice: }\PY{l+s+s1}{\PYZsq{}} \PY{o}{+} \PY{n+nb}{str}\PY{p}{(}\PY{n}{estimator\PYZus{}choice}\PY{p}{)}\PY{p}{)}
\end{Verbatim}


    \begin{Verbatim}[commandchars=\\\{\}]
estimator\_choice: 4500

    \end{Verbatim}

    \begin{Verbatim}[commandchars=\\\{\}]
{\color{incolor}In [{\color{incolor}9}]:} \PY{c+c1}{\PYZsh{}@title Training the Model with the Optimal Number of Estimators and Random States}
        \PY{n}{rf} \PY{o}{=} \PY{n}{RandomForestRegressor}\PY{p}{(}\PY{n}{n\PYZus{}estimators} \PY{o}{=} \PY{n}{estimator\PYZus{}choice}\PY{p}{,} \PY{n}{random\PYZus{}state} \PY{o}{=} \PY{n}{random\PYZus{}state\PYZus{}choice}\PY{p}{)}
        \PY{n}{rf}\PY{o}{.}\PY{n}{fit}\PY{p}{(}\PY{n}{x\PYZus{}train}\PY{p}{,} \PY{n}{y\PYZus{}train}\PY{p}{)}
        \PY{n}{y\PYZus{}pred} \PY{o}{=} \PY{n}{np}\PY{o}{.}\PY{n}{round}\PY{p}{(}\PY{n}{rf}\PY{o}{.}\PY{n}{predict}\PY{p}{(}\PY{n}{x\PYZus{}test}\PY{p}{)}\PY{p}{)}
        
        \PY{n}{macro} \PY{o}{=} \PY{n}{f1\PYZus{}score}\PY{p}{(}\PY{n}{y\PYZus{}test}\PY{p}{,} \PY{n}{y\PYZus{}pred}\PY{p}{,} \PY{n}{average}\PY{o}{=}\PY{l+s+s1}{\PYZsq{}}\PY{l+s+s1}{macro}\PY{l+s+s1}{\PYZsq{}}\PY{p}{)}
        \PY{n}{micro} \PY{o}{=} \PY{n}{f1\PYZus{}score}\PY{p}{(}\PY{n}{y\PYZus{}test}\PY{p}{,} \PY{n}{y\PYZus{}pred}\PY{p}{,} \PY{n}{average}\PY{o}{=}\PY{l+s+s1}{\PYZsq{}}\PY{l+s+s1}{micro}\PY{l+s+s1}{\PYZsq{}}\PY{p}{)}
        \PY{n}{weighted} \PY{o}{=} \PY{n}{f1\PYZus{}score}\PY{p}{(}\PY{n}{y\PYZus{}test}\PY{p}{,} \PY{n}{y\PYZus{}pred}\PY{p}{,} \PY{n}{average}\PY{o}{=}\PY{l+s+s1}{\PYZsq{}}\PY{l+s+s1}{weighted}\PY{l+s+s1}{\PYZsq{}}\PY{p}{)}
        
        \PY{n+nb}{print}\PY{p}{(}\PY{l+s+s2}{\PYZdq{}}\PY{l+s+s2}{Accuracy: }\PY{l+s+s2}{\PYZdq{}}\PY{p}{,} \PY{n}{accuracy\PYZus{}score}\PY{p}{(}\PY{n}{y\PYZus{}test}\PY{p}{,} \PY{n}{y\PYZus{}pred}\PY{p}{)}\PY{p}{)}
\end{Verbatim}


    \begin{Verbatim}[commandchars=\\\{\}]
Accuracy:  0.8541666666666666

    \end{Verbatim}

    \begin{Verbatim}[commandchars=\\\{\}]
{\color{incolor}In [{\color{incolor}10}]:} \PY{c+c1}{\PYZsh{}@title Confusion Matrix for the Random Forest Regressor.}
         \PY{n}{cm} \PY{o}{=} \PY{n}{confusion\PYZus{}matrix}\PY{p}{(}\PY{n}{y\PYZus{}test}\PY{p}{,} \PY{n}{y\PYZus{}pred}\PY{p}{)}
         \PY{n}{ax}\PY{o}{=} \PY{n}{plt}\PY{o}{.}\PY{n}{subplot}\PY{p}{(}\PY{p}{)}
         
         \PY{n}{sns}\PY{o}{.}\PY{n}{heatmap}\PY{p}{(}\PY{n}{cm}\PY{p}{,} \PY{n}{annot}\PY{o}{=}\PY{k+kc}{True}\PY{p}{,} \PY{n}{ax} \PY{o}{=} \PY{n}{ax}\PY{p}{)}
         
         \PY{n}{ax}\PY{o}{.}\PY{n}{set\PYZus{}title}\PY{p}{(}\PY{l+s+s2}{\PYZdq{}}\PY{l+s+s2}{Random Forest Classifier}\PY{l+s+s2}{\PYZdq{}}\PY{p}{)}
         \PY{n}{ax}\PY{o}{.}\PY{n}{set\PYZus{}xlabel}\PY{p}{(}\PY{l+s+s1}{\PYZsq{}}\PY{l+s+s1}{Predicted labels}\PY{l+s+s1}{\PYZsq{}}\PY{p}{)}
         \PY{n}{ax}\PY{o}{.}\PY{n}{set\PYZus{}ylabel}\PY{p}{(}\PY{l+s+s1}{\PYZsq{}}\PY{l+s+s1}{True labels}\PY{l+s+s1}{\PYZsq{}}\PY{p}{)}
         \PY{n}{ax}\PY{o}{.}\PY{n}{set\PYZus{}title}\PY{p}{(}\PY{l+s+s1}{\PYZsq{}}\PY{l+s+s1}{Confusion Matrix}\PY{l+s+s1}{\PYZsq{}}\PY{p}{)}
         \PY{n}{ax}\PY{o}{.}\PY{n}{xaxis}\PY{o}{.}\PY{n}{set\PYZus{}ticklabels}\PY{p}{(}\PY{p}{[}\PY{l+s+s1}{\PYZsq{}}\PY{l+s+s1}{Yes}\PY{l+s+s1}{\PYZsq{}}\PY{p}{,} \PY{l+s+s1}{\PYZsq{}}\PY{l+s+s1}{No}\PY{l+s+s1}{\PYZsq{}}\PY{p}{]}\PY{p}{)}
         \PY{n}{ax}\PY{o}{.}\PY{n}{yaxis}\PY{o}{.}\PY{n}{set\PYZus{}ticklabels}\PY{p}{(}\PY{p}{[}\PY{l+s+s1}{\PYZsq{}}\PY{l+s+s1}{Yes}\PY{l+s+s1}{\PYZsq{}}\PY{p}{,} \PY{l+s+s1}{\PYZsq{}}\PY{l+s+s1}{No}\PY{l+s+s1}{\PYZsq{}}\PY{p}{]}\PY{p}{)}
\end{Verbatim}


\begin{Verbatim}[commandchars=\\\{\}]
{\color{outcolor}Out[{\color{outcolor}10}]:} [Text(0, 0.5, 'Yes'), Text(0, 1.5, 'No')]
\end{Verbatim}
            
    \begin{center}
    \adjustimage{max size={0.9\linewidth}{0.9\paperheight}}{output_10_1.png}
    \end{center}
    { \hspace*{\fill} \\}
    
    Accuracy of the Model used: \textbf{0.854167}

\textbf{\emph{\textasciitilde{} 85.42\% accuracy}}


    % Add a bibliography block to the postdoc
    
    
    
    \end{document}
